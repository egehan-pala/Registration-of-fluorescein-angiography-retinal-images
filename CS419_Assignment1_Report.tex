\documentclass{article}
\usepackage{graphicx} % Required for inserting images

\title{CS 419 Assignment 1}
\author{Mehmet Egehan Pala 31089 }
\date{12 October 2024}

\begin{document}

\maketitle

\section{Tasks}
\subsection{First Step: Estimating the Transformation Matrix}
\quad In the assignment, first task is to estimate the Transformation Matrix by using the control points given in the file.
I have constructed a function using the linear algebra equation stated below as: \\

\begin{equation}
  \Large {A\cdot x = B}
\end{equation}\\

In basic 'A' is the source points that is created from the x-y control coordinates of test image stored in a $2\cdot N\times 6$ matrix form, and 'B' is the $2\cdot N\times 1$ vector that constructed from the reference image control points. In this equation 'x' is the translation matrix that should be found in order to make the affine translations. Furthermore, to find 'x' I have used "np.linalg.lstsq" which is a numpy library function uses the formula stated below to find unknown transformation matrix.\\

\begin{equation}
  \Large {\min_x \| A\cdot x - B \|_2^2}
\end{equation}\\
\begin{equation}
  \Large {x = (A^T\cdot A)^{-1} \cdot A^T\cdot B}
\end{equation}\\

By using the least squares equation, I have obtained the $2\times 3$ transformation matrix 'x', and later I have used the transformation matrix to apply transformations. This process have been repeated for both given image pairs.\\

\subsection{Second Step: Applying the Transformation}

\quad In the second step of the assignment, I have used "cv2.warpAffine" function from the computer vision library to apply affine transformations to the test images. In order to apply that line of code, I have constructed a function that takes an image and the transformation matrix, inverses the matrix and calculates the affine transformation. In order to invert the $2\times 3$ transformation matrix, I have added a new row of [0,0,1] resulted in a $3\times 3$ matrix which made it possible to invert the transformation matrix. Furthermore, after the inversion I have used "cv2.warpAffine" function and the linear algebra formula below to calculate the affine transformation. 


\begin{equation}
  \Large {x = A^{-1}\cdot B}
\end{equation}\\

Therefore, to ensure the alignment properly with the original image backward mapping is an essential step. On the other hand, I have calculated the Mean Square Error of the 2 images -the test image and the back transformed image- to calculate the error between the images in order to find out whether the matrix is estimated correctly.

\section{Error Calculation}
\quad In the final step of the assignment, I have calculated the Mean Square Error (MSE) between the test images and the back transformed images to find out error scale. For the first pair, the result was 1.17403 which is relatively low number stating nearly perfect alignment; however on the second pair the result of the MSE was higher with the score of 3.5686 which indicates a good alignment yet not perfect. 

\subsection{Several Factors that Might Cause Problems}

\subsubsection{Impact of Composition and Inverse Transformation Issues:}
\quad In the second pair of images test image is relatively brighter than the image generated by affine transformations. This could be the result of pixel mismatches during the transformation process especially pixel mismatching around the edges. The second pair of images consist neuron like structure which has lots of lines meaning that images consist lots of edges. The dark pixels around the edges meet with the brighter pixels where there will be high differences which contributes MSE to be higher.\\

\quad On the other hand, In the first pair of images, the test image is relatively darker then the image generated by affine transformations which could be explained by the same problem. 



\end{document}

